\PassOptionsToPackage{unicode=true}{hyperref} % options for packages loaded elsewhere
\PassOptionsToPackage{hyphens}{url}
%
\documentclass[]{article}
\usepackage{lmodern}
\usepackage{amssymb,amsmath}
\usepackage{ifxetex,ifluatex}
\usepackage{fixltx2e} % provides \textsubscript
\ifnum 0\ifxetex 1\fi\ifluatex 1\fi=0 % if pdftex
  \usepackage[T1]{fontenc}
  \usepackage[utf8]{inputenc}
  \usepackage{textcomp} % provides euro and other symbols
\else % if luatex or xelatex
  \usepackage{unicode-math}
  \defaultfontfeatures{Ligatures=TeX,Scale=MatchLowercase}
\fi
% use upquote if available, for straight quotes in verbatim environments
\IfFileExists{upquote.sty}{\usepackage{upquote}}{}
% use microtype if available
\IfFileExists{microtype.sty}{%
\usepackage[]{microtype}
\UseMicrotypeSet[protrusion]{basicmath} % disable protrusion for tt fonts
}{}
\IfFileExists{parskip.sty}{%
\usepackage{parskip}
}{% else
\setlength{\parindent}{0pt}
\setlength{\parskip}{6pt plus 2pt minus 1pt}
}
\usepackage{hyperref}
\hypersetup{
            pdftitle={Online Course Topics for Bayesian Modeling},
            pdfborder={0 0 0},
            breaklinks=true}
\urlstyle{same}  % don't use monospace font for urls
\usepackage[margin=1in]{geometry}
\usepackage{graphicx,grffile}
\makeatletter
\def\maxwidth{\ifdim\Gin@nat@width>\linewidth\linewidth\else\Gin@nat@width\fi}
\def\maxheight{\ifdim\Gin@nat@height>\textheight\textheight\else\Gin@nat@height\fi}
\makeatother
% Scale images if necessary, so that they will not overflow the page
% margins by default, and it is still possible to overwrite the defaults
% using explicit options in \includegraphics[width, height, ...]{}
\setkeys{Gin}{width=\maxwidth,height=\maxheight,keepaspectratio}
\setlength{\emergencystretch}{3em}  % prevent overfull lines
\providecommand{\tightlist}{%
  \setlength{\itemsep}{0pt}\setlength{\parskip}{0pt}}
\setcounter{secnumdepth}{0}
% Redefines (sub)paragraphs to behave more like sections
\ifx\paragraph\undefined\else
\let\oldparagraph\paragraph
\renewcommand{\paragraph}[1]{\oldparagraph{#1}\mbox{}}
\fi
\ifx\subparagraph\undefined\else
\let\oldsubparagraph\subparagraph
\renewcommand{\subparagraph}[1]{\oldsubparagraph{#1}\mbox{}}
\fi

% set default figure placement to htbp
\makeatletter
\def\fps@figure{htbp}
\makeatother


\title{Online Course Topics for Bayesian Modeling}
\author{}
\date{\vspace{-2.5em}}

\begin{document}
\maketitle

\hypertarget{general}{%
\subsection{General}\label{general}}

This website is for hosting material related to Bayesian modeling,
Generalised Additive Models (GAMs), the statistical tool R-INLA, the
SPDE approach, and my own research.

This website is always under development. Recent changes can be found at
\href{./feedback.html}{Feedback and changes}

The website was recently moved from Bitbucket to Github (Dec 2019). This
was mainly because Bitbucket support is ending for hg, and I like
Githubs git interface.

\hypertarget{warning-on-r-versions}{%
\subsection{Warning on R versions}\label{warning-on-r-versions}}

I recently updated to R 3.6, and had to change some of the scripts
because of this.

\hypertarget{the-.r-files}{%
\subsection{The .R files}\label{the-.r-files}}

In any BTopic, go to the web-page address (e.g.~command+l), replace the
.html with .R, and this should give the same R file. It is a lot easier
to get the code this way than to copy-paste line by line!

\hypertarget{what-can-you-use-this-webpage-for}{%
\subsection{What can you use this webpage
for?}\label{what-can-you-use-this-webpage-for}}

I get questions like: Can I use this for

\begin{itemize}
\tightlist
\item
  Teaching?
\item
  Send it to others?
\item
  As a basis for my own code?
\item
  Include code in my presentations?
\end{itemize}

The answers to all of these is ``yes''. Please add an acknowledgement of
this webpage when you do so.

However, if you are inspired to write papers based on some of the ideas
you see here, I may already be doing that. I would be happy to hear from
you, and I can clarify if we are working on the same idea.

\hypertarget{online-offline-use-and-older-versions}{%
\subsection{Online / Offline use and older
versions}\label{online-offline-use-and-older-versions}}

The online webpage is at {[}\url{https://haakonbakkagit.github.io/}{]}.

To use this webpage offline, download the github repository at
{[}\url{https://github.com/haakonbakkagit/haakonbakkagit.github.io}{]},
open the folder and click on index.html.

To find older versions of the code for any reason, use the same link and
look through the history of the repo.

\hypertarget{philosophy-of-this-website}{%
\subsection{Philosophy (of this
website)}\label{philosophy-of-this-website}}

The explanations in this website are through pictures, figures and
computer code.

The text on the website is informal, trying to paint a picture and give
the right notions, ideas and intuition. If you want rigorous
mathematics, see the papers that are referenced/linked throughout the
website. The goal of a paper is usually to be precise and impressive,
but the goal of these pages is to make things look simple and to speak
to your intuition.

Why am I using html and not pdf? Because I want to use hyperlinks! I
want you to be able to open new tabs, and use the ``back'' arrow in your
browser. To open a figure in a new tab (and you can look at two figures
side by side in two windows). I want to integrate other websites and
content, e.g.~youtube videos.

\hypertarget{plan}{%
\subsection{Plan}\label{plan}}

Plan for the site: Add topics as requested by you, and to fill the gaps
in the topic organisation.

\end{document}
